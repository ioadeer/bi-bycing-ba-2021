
%%%%%%%%%%%%%%%%%%%%%%% file typeinst.tex %%%%%%%%%%%%%%%%%%%%%%%%%
%
% This is the LaTeX source for the instructions to authors using
% the LaTeX document class 'llncs.cls' for contributions to
% the Lecture Notes in Computer Sciences series.
% http://www.springer.com/lncs       Springer Heidelberg 2006/05/04
%
% It may be used as a template for your own input - copy it
% to a new file with a new name and use it as the basis
% for your article.
%
% NB: the document class 'llncs' has its own and detailed documentation, see
% ftp://ftp.springer.de/data/pubftp/pub/tex/latex/llncs/latex2e/llncsdoc.pdf
%
%%%%%%%%%%%%%%%%%%%%%%%%%%%%%%%%%%%%%%%%%%%%%%%%%%%%%%%%%%%%%%%%%%%


\documentclass[runningheads,a4paper,spanish]{llncs}

\usepackage[utf8]{inputenc}
\usepackage[spanish]{babel}
\usepackage{amssymb}
\setcounter{tocdepth}{3}
\usepackage{graphicx}

\usepackage{url}
\urldef{\mailsa}\path|{lifofernandez, jenniffer.chavez, cjoackin}@gmail.com|    
\newcommand{\keywords}[1]{\par\addvspace\baselineskip
\noindent\keywordname\enspace\ignorespaces#1}

\begin{document}

\mainmatter  % start of an individual contribution

% first the title is needed
\title{Bussiness Intelligence aplicado al servicio de bicing de la Ciudad Autónoma de Buenos Aires}

% a short form should be given in case it is too long for the running head
\titlerunning{BI aplicado al servicio de bicing de CABA}

% the name(s) of the author(s) follow(s) next
%
% NB: Chinese authors should write their first names(s) in front of
% their surnames. This ensures that the names appear correctly in
% the running heads and the author index.
%
\author{Alfred Hofmann%
\thanks{Please note that the LNCS Editorial assumes that all authors have used
the western naming convention, with given names preceding surnames. This determines
the structure of the names in the running heads and the author index.}%
\and Ursula Barth\and Ingrid Haas\and Frank Holzwarth\and\\
Anna Kramer\and Leonie Kunz\and Christine Rei\ss\and\\
Nicole Sator\and Erika Siebert-Cole\and Peter Stra\ss er}
%
\authorrunning{Lecture Notes in Computer Science: Authors' Instructions}
% (feature abused for this document to repeat the title also on left hand pages)

% the affiliations are given next; don't give your e-mail address
% unless you accept that it will be published
\institute{Springer-Verlag, Computer Science Editorial,\\
Tiergartenstr. 17, 69121 Heidelberg, Germany\\
\mailsa\\
\url{https://github.com/ioadeer/bi-bycing-ba-2021}}

\toctitle{Lecture Notes in Computer Science}
\tocauthor{Authors' Instructions}
\maketitle

\begin{abstract}
Este es un estudio preliminar de los viajes realizados en el sistema público de
bicicletas de la Ciudad de Buenos Aires durante todo el año 2020 y el primer
semestre de 2021.
Se prepararon más de 3 millones de registros de viajes y se adquirieron más de
13000 datos climáticos horarios. Se comprobó la incidencia
del estado del clima, emparejando dichos recorridos a las observaciones
horarias meteorológicas para el mismo período de tiempo.
Se estudió el efecto modulador periodico semanal de la estacionalidad días
hábiles- fines de semana en uso del servicio además de los acontecimientos/
particularidades eventuales del tipo festividades, feriados nacionales o días
no laborales.
Se produjeron subconjuntos de datos derivados de contabilizar las salidas,
arribos y usuarios;  Se descubrieron características a partir de la relación
entre esto y la capacidad de cada estación.
Otras características que se desarrollaron son las del tipo de recorrido, ida
de una estación a otra o bien, vuelta a la misma estación; El cálculo de la
distancia del viaje derivó en la posibilidad de establecer la velocidad de
desplazamiento de los usuarios y a partir de ello desarrollar múltiples
características.
Establecer la relación entre el origen y destino de cada viaje a los
información de cada estación habilitó estructurar en nubes de puntos espaciales
lo que permitió analizar diversas propiedades, para múltiples ventanas de
tiempo con diferentes frecuencias de muestreo.  Se desarrolló un modelo basado
en aprendizaje automático (ML) de predicción de demanda y prescripción de
suministro para cada estación dado el input del dato del tiempo.
\keywords{
regeneración digitalmente inclusiva, servicios electrónicos,
códigos de diseño basados en formularios, gobernanza, ciudades inteligentes,
trayectorias de conocimiento, planes maestros, redes, 
transición
}
\end{abstract}

\section{Introdución}

Ecobici es el sistema de transporte público de
bicicletas compartidas que desarrolló el Gobierno de
la Ciudad de Buenos Aires, brindando a quienes se
mueven en la Ciudad una nueva alternativa de
transporte.  La operación de BA Ecobici es gestionada
por Tembici, una startup brasileña de movilidad
urbana.
Desde sus inicios el sistema fue operado por el
Gobierno de la Ciudad de Buenos Aires.  En marzo de
2018, la Legislatura de la Ciudad de Buenos Aires
aprobó un proyecto de ley para que el sistema sea
gestionado por un tercero, que se hará cargo por 10
años junto a su modernización, operación,
mantenimiento y explotación. Así, el Estado se hace
cargo del 22% del costo total del servicio para
mantener la gratuidad, y el resto por la
publicidad.11 En julio de ese año resultó
ganadora de la licitación pública la empresa brasileña
Tembici que es la encargada de explotar el sistema a
partir de 2019.12 El nuevo sistema duplicará la
cantidad de estaciones y bicicletas disponibles.
Hasta ese momento el sistema contaba con 4.000
bicicletas y 400 estaciones.  En enero de 2019 comenzó
el traspaso al nuevo sistema por lo que se cerraron 54
estaciones para su renovación en la primera etapa, con
un gran descontento.13 El 25 de febrero de
201914la empresa brasileña Tembici se hizo cargo
del servicio. Con una inversión de 70 millones de
dólares15, renovaron todas las bicicletas y las
estaciones.  A fines de junio de 2019, se cumplió la
meta de 400 estaciones con 4000 rodados, según fuentes
oficiales.16 Sin embargo, a partir del segundo
semestre de 2019 el número de bicicletas disponibles
empezó a disminuir en forma constante. También bajó el
número de bicicletas bloqueadas. Para enero de
202017 quedaban menos de mil rodados, y la
respuesta del gobierno porteño fue empezar a cerrar
estaciones por "baja demanda".  A fines de febrero de
2020, un año después del arribo de Tembici, quedaban
alrededor de 300 bicicletas disponibles para 239.000
usuarios. Dada la improbabilidad de conseguir una
bicicleta, su uso era prácticamente nulo.  El 20 de
marzo de 2020 el sistema fue suspendido por la
cuarentena decretada por el Gobierno Nacional.  En
mayo de 2020 el sistema fue restablecido a la mitad de
lo que debería funcionar: 200 estaciones y 1000
bicicletas. El tiempo de uso gratuito fue reducido de
una hora los días de semana y dos horas los sábados y
domingos a 30 minutos, mientras que el tiempo de
espera para sacar una nueva bicicleta se incrementó de
5 minutos a 15.  En marzo de 2021, la gratuidad del
sistema fue reducida drásticamente a únicamente cuatro
viajes de 30 minutos por día, en días hábiles. El
costo de un viaje de media hora fuera de este rango
triplicaría el del transporte público. También se
agregaron varias opciones de pases diarios, mensuales
y anuales, con precios diferenciados para locales y
turistas, todos con fuertes restricciones a la
cantidad y duración de los viajes incluidos. 

\subsection{Checking the PDF File}


\newpage
\bibliography{referencias.bib}
\bibliographystyle{plain}

\end{document}
